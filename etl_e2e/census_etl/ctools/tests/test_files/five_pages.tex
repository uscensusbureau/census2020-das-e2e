\documentclass{article}
\begin{document}
page 1 \clearpage
page 2 \clearpage
page 3 \clearpage
page 4 \clearpage
page 5 \clearpage
\end{document}
